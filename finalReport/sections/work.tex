This work makes use of probability meshes to approximate a 4-dimensional (position,velocity) space. The meshes are then used to implement efficient forward estimation, integration of new sensor data, and integration of probabilistic rules. Finally, a pipeline is constructed from these elements to produce refined paths from raw positional input.

\section{Probability Meshes}

Each trajectory of interest $t$ has an associated 4-dimensional $n*m*i*j$ probability mesh $M_t$, with each cell $M_t(n,m,i,j)$ in the mesh representing the likelihood that the most recent position and velocity in the trajectory lie in that cell.
However $n$, $m$, $i$, and $j$ are integer values while locations and velocities are continuous.
In order to account for continuous values, an interpolation scheme can approximate points between cell centers.

Interpolation works as follows: For each dimension index $i$ in a query $M_t(i_n, i_m, i_i, i_j)$, calculate $\mathit{floor}(i)$ and $\mathit{ceil}(i)$. Then collect the set of points $P$ for all combinations of $floor$ and $ceil$ in each dimension.

For each point $p \in P$,

\taylor{Detail problem of discretizing continuous probs}
\taylor{cite original 2d grid work}
\taylor{discuss extension to 4d to include velocity}

\taylor{Discuss 2-way mapping from grid to real-world}
\taylor{Discuss implementation of linear interpolation}
\taylor{Discuss benefits/drawbacks of linear interpolation=>max/min at point}

\section{Probabilistic Rules}

\taylor{Discuss ability to represent knowledge of objects being tracked}
\taylor{Contrast to Kalman Filters, which can only represent Gaussians}
\taylor{Provide examples: Vehicles drive on roads, in the correct direction, people near their home not moving are in their homes}
\taylor{Emphasize probability to allow rules that are sometimes wrong}

\subsection{Road Matching}

\taylor{Discuss and cite data source (OSM)}

\taylor{Give function over minimum road distance}

\taylor{Efficiently calculating minimum road distance}
\taylor{discuss roads: sequence of segments}
\taylor{discuss index implementation - each road in each grid cell it crosses}
\taylor{discuss index query - need query point + neighbors (nearest q, not overlap q)}

\section{Grid-Based Trajectory Refinement}

\taylor{summarize how grids + rules can combine information}

\subsection{Forward Estimation}

\taylor{Describe example point in grid with velocity moving forward}
\taylor{provide math for pointwise update}

\subsection{GPS Sensor Integration}

\taylor{Describe sensor input, gaussian centered around point}
\taylor{Cite error distribution again}
\taylor{provide math for pointwise integration}
\taylor{Call out mult as pointwise AND}

\subsection{Rule Integration}

\taylor{Call back to sensor integration}
\taylor{Note equation compatibility with any probability rule}
\taylor{Summarize use of OSM Road Matching}
